% !Mode:: "TeX:UTF-8"
\chapter{总结与展望}
\label{conclusion}

\section{全文总结}
\label{summary}
本工作提出了一个双路物体检测与跟踪(Dual-way Object Detection and Tracking, DODT)框架, 旨在将三维物体检测从单帧推广到多帧连续场景,从而推动前沿三维物体检测算法在自动驾驶领域的落地。DODT的主要思想是基于流数据的连续性以及冗余性,通过只对关键帧做物体检测,然后在时序信息的引导下将关键帧检测结果传播到非关键帧,最后再将帧间数据关联,完成三维物体检测与多目标追踪任务。在该思想的引导下,本文分了四部分详细介绍了DODT的研究背景、研究基础、原理实现以及实验验证。在第一章中,本文详细阐述了近年来国内外在三维目标检测领域的进展,并分析了各流派的优缺点,以及将单帧方法推广到多帧流数据场景的重要意义。第二章则详细介绍了基于深度学习技术的目标检测技术进展,重点介绍了以Faster-RCNN为代表的两阶段目标检测以及以YOLO为代表的单阶段目标检测的原理和实现方式,为读者了解DODT的原理提供技术参考。之后本章还简要介绍了单目标追踪的方法原理,以及多目标追踪的研究进展以及性能衡量指标。本文第三章开始重点介绍了DODT的网络架构与实现原理,先后详细介绍了组成DODT的四个基本模块:三维物体检测模块、\textit{Shared RPN}模块、时序信息处理模块和运动插值模块,以及这些模块如何相互配合完成流数据的物体检测与多目标跟踪任务。第四章阐述了DODT的实验验证环节,该章首先介绍了实验所用的KITTI数据集以及数据预处理步骤,然后使用控制变量法分析了DODT个模块对最终检测与追踪性能的影响,并分析出了最佳的关键帧选取步长。另外,该章还介绍了DODT在三维多目标追踪领域与前沿方法的性能对比,实验结果表明DODT能够取得与前沿方法匹敌的性能,并且有着自己的独特优势。


\section{展望}
\label{future}
DODT虽然在流数据物体检测与跟踪任务取得了显著的效果,但它离运用到自动驾驶平台还有很长一段距离。DODT框架的落地还需要解决以下四个问题:
\begin{itemize}
\item 目前的检测模型只局限于AVOD网路,能否将其扩展到任何三维物体检测模型,例如目前PointRCNN系列检测框架?
\item 目前DODT是以近似在线跟踪的方式实现多目标跟踪,能否在DODT的基础上设计出在线的三维多目标跟踪算法?
\item DODT模型在长关键帧选取步长上效果不佳,是否有更为有效的关键帧选取算法?以及是否有更加高效的时序信息提取方法?
\item DODT框架要落地还需进一步提升速度,能否使用模型压缩方法进一步提高模型的运行效率,使其能够迁移到嵌入式设备上?
\end{itemize}
本项目后续工作将重点从这四个方面入手,继续改进现有的DODT框架。对于第一个问题,设计适应性更广、扩展性更强的DODT框架,目前我们已开展相关工作。由于三维物体检测领域算法日新月异,目前KITTI排行榜的前几位算法都是只基于点云数据的,并且也有先进行点云分割再回归目标框的算法。因此扩展版的DODT应将这些算法囊括进来,构建一个通用的三维流数据物体检测框架。在该基础上,在线多目标跟踪算法、关键帧选取算法以及更加高效的时序信息提取算法的探索可以同步进行。DODT的嵌入式迁移是本项目最后的工作,目前我们考虑了参数压缩以及二值化网络的方案。只有将DODT在嵌入式设备上运行,才能够真正意义上实现算法落地,推进自动驾驶领域的技术革新。

% 打印时插入必要的空白页
\ifprint
	\newpage
	\thispagestyle{empty}
	\mbox{}
	
	% 避免空白页影响页码编号
	\clearpage
	\setcounter{page}{10}
\fi