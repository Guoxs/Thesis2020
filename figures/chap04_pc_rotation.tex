\begin{figure}[t]
	\begin{center}
		\begin{overpic}[width=\textwidth]{imgs/pc_before_rotation.png}
			\put(5, 5){\color{white}{校准前}}
		\end{overpic}\vspace{4pt}
		\begin{overpic}[width=\textwidth]{imgs/pc_after_rotation.png}
			\put(5, 5){\color{white}{校准后}}
		\end{overpic}
	\end{center}
	\vspace{-0.8cm}
	\caption{两帧相邻关键帧点云数据校准到同一个坐标系。其中白色框为前一帧关键帧标签,绿色框为后一帧关键帧标签,框上的数字为框标号。可以看出,校准前两帧点云中的同一物体偏离很大,无法有效融合两帧数据;而校准后两帧中同一物体基本重合在一起,些许偏移为物体运动造成的。校准后的数学可以有效融合,也更容易提取时序特征。}
	\label{fig:pc_rotation}
\end{figure}

