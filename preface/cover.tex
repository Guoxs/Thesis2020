% !Mode:: "TeX:UTF-8"

\cheading{中山大学硕士学位论文}      % 设置正文的页眉,需要填上对应的毕业年份
\ctitle{基于流数据的三维物体检测与追踪}    % 封面用论文标题,自己可手动断行
\etitle{3D Streaming-based Object Detection and Tracking}    %论文英文标题
\csubject{计算机科学与技术}   % 专业名称
\esubject{Computer Science and Technology}


\ifreview
	% 盲审时用
	\cauthor{ }            % 学生姓名
	\eauthor{ }
	\csupervisor{ }        % 导师姓名
	\esupervisor{ }
\else
	\cauthor{郭叙森}            % 学生姓名
	\eauthor{Guo Xusen}
	\csupervisor{黄凯~教授}        % 导师姓名,~用于间隔职称
	\esupervisor{Prof. Huang Kai}
\fi


% 自动数字日期
%\cdate{\the\year~年~\the\month~月~\the\day~日}
% 自动中文日期
%\cdate{\CJKnumber{\the\year}~年~\CJKnumber{\the\month}~月~\CJKnumber{\the\day}~日}
% 定制中文日期
%\cdate{二零二零~年~五~月~二十~日}
\cdate{二〇二〇~年~五~月~二十~日}

\cabstract{
近年来,随着深度学习技术的不断发展,三维物体检测领域取得了许多令人瞩目的成果,这些研究成果带动着众多自动驾驶创业公司蓬勃发展。然而,目前大多数三维物体检测方法都是针对单帧数据的,这些方法忽略了帧数据的时间连续性,无法利用帧与帧之间的时序信息,对检测算法的落地增加了不少难度。本工作旨在探索三维物体检测中时序信息的利用问题,通过深度学习技术去挖掘流数据的帧间相关性。为此,本文针对性地构建了一个基于关键帧的双路三维物体检测网络。该网络首先对关键帧进行物体检测,然后基于神经网络学习到的时序信息,将关键帧的检测结果传播到非关键帧。该传播过程可以将不同帧的同一物体关联,因此在得到检测结果之后,追踪结果也很容易获得,从而使得该网络同时具备三维物体检测与多目标追踪的能力。实验表明,本方法在速度与精度上都要优于同类型的单帧物体检测网络。在KITTI数据集的目标追踪比赛中,本方法在三维多目标追踪任务中取得了和当前最好方法相当的性能,分别取得了76.68\%的MOTA和81.65\%的MOTP。
}

\ckeywords{自动驾驶;三维物体检测;多目标追踪;时序信息;关键帧}

\eabstract{
Recent approaches for 3D object detection have made tremendous progresses due to the development of deep learning. However, previous researches for detection are mostly based on individual frames, leading to limited exploitation of information between frames. In this paper, we attempt to leverage the temporal information in streaming data and explore 3D streaming based object detection as well as tracking. Toward
this goal, we set up a dual-way network for 3D object detection based on key frames, then propagate predictions to non-key frames through a motion based interpolation algorithm guided by temporal information. Our framework is not only shown to have significant improvements on object detection compared with frame-by-frame paradigm, but also proven to produce competitive results on KITTI Object Tracking Benchmark, with
76.68\% in MOTA and 81.65\% in MOTP respectively.
}

\ekeywords{Autonomous driving; 3D object detection; Multiple object tracking; Temporal information; Key frame}

\makecover

\clearpage
